\documentclass{beamer}

\usepackage[english,russian]{babel}
\usepackage[utf8]{inputenc}
\usepackage[all]{xy}
\usepackage{ifthen}
\usepackage{xargs}

\usetheme{Szeged}
% \usetheme{Montpellier}
% \usetheme{Malmoe}
% \usetheme{Berkeley}
% \usetheme{Hannover}
\usecolortheme{beaver}

\newcommand{\newref}[4][]{
\ifthenelse{\equal{#1}{}}{\newtheorem{h#2}[hthm]{#4}}{\newtheorem{h#2}{#4}[#1]}
\expandafter\newcommand\csname r#2\endcsname[1]{#4~\ref{#2:##1}}
\newenvironmentx{#2}[2][1=,2=]{
\ifthenelse{\equal{##2}{}}{\begin{h#2}}{\begin{h#2}[##2]}
\ifthenelse{\equal{##1}{}}{}{\label{#2:##1}}
}{\end{h#2}}
}

\newref[section]{thm}{theorem}{Theorem}
\newref{lem}{lemma}{Lemma}
\newref{prop}{proposition}{Proposition}
\newref{cor}{corollary}{Corollary}

\theoremstyle{definition}
\newref{defn}{definition}{Definition}

\newcommand{\cat}[1]{\mathbf{#1}}
\renewcommand{\C}{\cat{C}}
\newcommand{\Set}{\cat{Set}}
\newcommand{\Grp}{\cat{Grp}}
\newcommand{\Ab}{\cat{Ab}}
\newcommand{\Ring}{\cat{Ring}}
\renewcommand{\Vec}{\cat{Vec}}
\newcommand{\Hask}{\cat{Hask}}
\newcommand{\Mat}{\cat{Mat}}
\newcommand{\Num}{\cat{Num}}

\newcommand{\im}{\mathrm{Im}}
\newcommand{\bool}{\mathrm{Bool}}
\newcommand{\true}{\mathrm{true}}
\newcommand{\false}{\mathrm{false}}

\newcommand{\pb}[1][dr]{\save*!/#1-1.2pc/#1:(-1,1)@^{|-}\restore}
\newcommand{\po}[1][dr]{\save*!/#1+1.2pc/#1:(1,-1)@^{|-}\restore}

\AtBeginSection[]
{
\begin{frame}[c,plain,noframenumbering]
\frametitle{План лекции}
\tableofcontents[currentsection]
\end{frame}
}

\makeatletter
\defbeamertemplate*{footline}{my theme}{
    \leavevmode
}
\makeatother

\begin{document}

\title{Теория категорий}
\subtitle{Пределы и копределы}
\author{Валерий Исаев}
\maketitle

\section{Уравнители}

\begin{frame}
\frametitle{Уравнители в $\Set$}
\begin{itemize}
\item Часто новые множества конструируются из уже существующих как подмножества элементов, удовлетворяющих некоторому уравнению.
\item Например, множество неотрицательных вещественных чисел $\mathbb{R}_{\geq 0}$ является подмножеством $\mathbb{R}$ таких $x$, что $|x| = x$.
\item Другой пример: множество корней полинома $p$ является подмножеством $\mathbb{R}$ таких $x$, что $p(x) = 0$.
\item В общем случае, если $f, g : A \to B$ -- пара функций, то \emph{уравнитель} этих функций -- это подмножество $A$ таких $x$, что $f(x) = g(x)$.
\end{itemize}
\end{frame}

\begin{frame}
\frametitle{Уравнители в произвольной категории}
\emph{Уравнитель} пары морфизмов $f, g : A \to B$ -- это мономорфизм $e : E \to A$ такой, что $f \circ e = g \circ e$ и для любого $h : F \to A$ такого, что $f \circ h = g \circ h$, существует стрелка $k : F \to E$ такая, что $e \circ k = h$.
\[ \xymatrix{ F \ar@{-->}[d]_{\exists k} \ar[dr]^h \\
              E \ar@{^{(}->}[r]_e & A \ar@<-.5ex>[r]_g \ar@<.5ex>[r]^f & B
            } \]

Мономорфизм называется \emph{регулярным}, если он является уравнителем некоторой пары стрелок.
\end{frame}

\begin{frame}
\frametitle{Другое определение уравнителей}
\emph{Уравнитель} пары морфизмов $f, g : A \to B$ -- это морфизм $e : E \to A$ такой, что $f \circ e = g \circ e$ и для любого $h : F \to A$ такого, что $f \circ h = g \circ h$, существует уникальная стрелка $k : F \to E$ такая, что $e \circ k = h$.
\[ \xymatrix{ F \ar@{-->}[d]_{\exists ! k} \ar[dr]^h \\
              E \ar[r]_e & A \ar@<-.5ex>[r]_g \ar@<.5ex>[r]^f & B
            } \]

Упражнение: докажите, что эти определения эквивалентны.
\end{frame}

\begin{frame}
\frametitle{Уравнители в категории $\Ab$}
\begin{itemize}
\item В категории $\Ab$ абелевых групп уравнители тесно связаны с понятием ядра морфизма.
\item Если $f : A \to B$ -- морфизм абелевых групп, то ядро $f$ -- это уравнитель пары стрелок $f, 0 : A \to B$.
\item И наоборот, если $f, g : A \to B$ -- пара морфизмов, то их уравнитель -- это ядро морфизма $f - g : A \to B$.
\item Таким образом, в категории $\Ab$ существуют все уравнители.
\end{itemize}
\end{frame}

\section{Пределы}

\begin{frame}
\frametitle{Конусы диграмм}
\begin{itemize}
\item Пусть $J = (V,E)$ -- некоторый граф, и $D$ -- диграмма формы $J$ в категории $\C$.
\item \emph{Конус} диаграммы $D$ -- это объект $A$ вместе с коллекцией морфизмов $a_v : A \to D(v)$ для каждой $v \in V$, удовлетворяющие условию, что для любого $e \in E$ следующая диаграмма коммутирует
\[ \xymatrix{ A \ar[d]_{a_{s(e)}} \ar[rd]^{a_{t(e)}} \\
              D(s(e)) \ar[r]_{D(e)} & D(t(e))
            } \]
\end{itemize}
\end{frame}

\begin{frame}
\frametitle{Определение пределов}
\begin{itemize}
\item \emph{Предел} диграммы $D$ -- это такой конус $A$, что для любого конуса $B$ существует уникальный морфизм $f : B \to A$, такой что для любой $v \in V$ следующая диаграмма коммутирует
\[ \xymatrix{ B \ar[d]_{b_v} \ar[r]^f & A \ar[dl]^{a_v} \\
              D(v)
            } \]
\item Предел $D$ обозначается $lim\ D$.
\item Категория называется \emph{полной} (\emph{конечно полной}), если в ней существуют все малые (конечные) пределы.
\end{itemize}
\end{frame}

\begin{frame}
\frametitle{Примеры пределов}
\begin{itemize}
\item Произведения -- это пределы дискретных диаграмм.
\item Бинарные произведения -- это пределы диаграмм вида \[ \bullet \qquad \bullet \]
\item Уравнители -- это пределы диаграмм вида
\[ \xymatrix{ \bullet \ar@<-.5ex>[r] \ar@<.5ex>[r] & \bullet } \]
\item Терминальные объекты -- это пределы пустой диаграммы.
\end{itemize}
\end{frame}

\begin{frame}
\frametitle{Уникальность пределов}
\begin{prop}
Если $A$ и $B$ -- пределы диаграммы $D$, то существует изоморфизм $f : A \simeq B$, такой что $a_v = b_v \circ f$ для любой $v \in V$.
\end{prop}
\begin{proof}
Так как $B$ -- предел, то существует стрелка $f : A \to B$, удовлетворяющая условию утверждения.
Так как $A$ -- предел, то существует стрелка $g : B \to A$.
По уникальности мы знаем, что $g \circ f = id_A$ и $f \circ g = id_B$, то есть $f$ -- изоморфизм.
\end{proof}
\end{frame}

\begin{frame}
\frametitle{Пулбэки}
\begin{itemize}
\item \emph{Пулбэки} -- это пределы диаграмм вида
\[ \xymatrix{                & \bullet \ar[d] \\
              \bullet \ar[r] & \bullet
            } \]
\item Пулбэк можно изображать как коммутативный квадрат
\[ \xymatrix{ A \times_C B \ar[d] \ar[r] \pb & B \ar[d] \\
              A \ar[r]                       & C
            } \]
\item Пулбэк иногда называют декартовым квадратом.
\item Стрелку $A \times_C B \to A$ называют пулбэком стрелки $B \to C$.
\end{itemize}
\end{frame}

\begin{frame}
\frametitle{Декартово произведение через пулбэки}
\begin{prop}
Если $1$ -- терминальный объект, то пулбэк $A \times_1 B$ является декартовым произведением $A \times B$.
\end{prop}
\begin{proof}
Действительно, конус диаграммы $\quad A \qquad B \quad$ -- это тоже самое, что и конус диаграммы 
\[ \xymatrix{          & B \ar[d] \\
              A \ar[r] & 1
            } \]
Следовательно пределы этих диграмм также совпадают.
\end{proof}
\end{frame}

\begin{frame}
\frametitle{Пулбэки в $\Set$}
В $\Set$ пулбэк диаграммы
\[ \xymatrix{          & B \ar[d]^g \\
              A \ar[r]_f & C
            } \]
можно определить как подмножество декартова произведения $A \times B$.
Действительно, если мы положим $A \times_C B = \{ (a, b)\ |\ f(a) = g(b) \}$, то легко видеть, что $A \times_C B$ является пулбэком диграммы выше.
\end{frame}

\begin{frame}
\frametitle{Пулбэки через уравнители и произведения}
\begin{prop}
Если в категории существуют конечные произведения и уравнители, то в ней существуют пулбэки.
\end{prop}
\begin{proof}
Пулбэки можно сконструировать так же, как и в $\Set$.
Пусть $e : D \to A \times B$ -- уравнитель стрелок $f \circ \pi_1 : A \times B \to C$ и $g \circ \pi_2 : A \times B \to C$.
Тогда легко видеть, что квадрат ниже является декартовым.
\[ \xymatrix{ D \ar[d]_{\pi_1 \circ e} \ar[r]^{\pi_2 \circ e} & B \ar[d]^g \\
              A \ar[r]_f                                      & C
            } \]
\end{proof}
\end{frame}

\begin{frame}
\frametitle{Пределы через уравнители и произведения}
\begin{prop}
Если в категории существуют конечные произведения и уравнители, то в ней существуют все конечные пределы.
\end{prop}
\begin{proof}
Пусть $D$ -- диаграмма формы $(V,E)$. Тогда рассмотрим диаграмму, состоящую из пары стрелок
\[ \langle \pi_{t(e)} \rangle_{e \in E}, \langle D(e) \circ \pi_{s(e)} \rangle_{e \in E} : \prod_{v \in V} D(v) \rightrightarrows \prod_{e \in E} D(t(e)) \]
Конус этой диаграммы -- это тоже самое, что конус диаграммы $D$.
Следовательно предел этой диаграммы также является пределом $D$.
\end{proof}
\end{frame}

\begin{frame}
\frametitle{Прообраз подобъекта}
\begin{itemize}
\item Пусть $f : A \to C$ -- функция в $\Set$ и $B \subseteq C$.
\item Тогда мы можем определить прообраз $f$: $f^{-1}(B) = \{ a \in A\ |\ f(a) \in B \} \subseteq A$.
\item Как обобщить эту конструкцию на произвольную категорию?
\item \emph{Прообраз} подобъекта $B \hookrightarrow C$ вдоль морфизма $f : A \to C$ -- это пулбэк
\[ \xymatrix{ f^{-1}(B) \ar@{^{(}->}[d] \ar[r] \pb & B \ar@{^{(}->}[d] \\
              A \ar[r]_f                           & C
            } \]
\item Упражнение: докажите, что $f^{-1}(B) \to A$ является мономорфизмом.
\end{itemize}
\end{frame}

\begin{frame}
\frametitle{Пересечение подобъектов}
\begin{itemize}
\item Пусть $A$ и $B$ -- подмножества $C$.
\item Тогда мы можем определить их пересечение $A \cap B$, которое является подмножеством и $A$, и $B$.
\item Как обобщить эту конструкцию на произвольную категорию?
\item \emph{Пересечение} подобъектов $A \hookrightarrow C$ и $B \hookrightarrow C$ -- это пулбэк
\[ \xymatrix{ A \cap B \ar@{^{(}->}[d] \ar@{^{(}->}[r] \pb & B \ar@{^{(}->}[d] \\
              A \ar@{^{(}->}[r]                             & C
            } \]
\end{itemize}
\end{frame}

\end{document}
