\documentclass{beamer}

\usepackage[english,russian]{babel}
\usepackage[utf8]{inputenc}
\usepackage[all]{xy}
\usepackage{ifthen}
\usepackage{xargs}

\usetheme{Szeged}
% \usetheme{Montpellier}
% \usetheme{Malmoe}
% \usetheme{Berkeley}
% \usetheme{Hannover}
\usecolortheme{beaver}

\newcommand{\newref}[4][]{
\ifthenelse{\equal{#1}{}}{\newtheorem{h#2}[hthm]{#4}}{\newtheorem{h#2}{#4}[#1]}
\expandafter\newcommand\csname r#2\endcsname[1]{#4~\ref{#2:##1}}
\newenvironmentx{#2}[2][1=,2=]{
\ifthenelse{\equal{##2}{}}{\begin{h#2}}{\begin{h#2}[##2]}
\ifthenelse{\equal{##1}{}}{}{\label{#2:##1}}
}{\end{h#2}}
}

\newref[section]{thm}{theorem}{Theorem}
\newref{lem}{lemma}{Lemma}
\newref{prop}{proposition}{Proposition}
\newref{cor}{corollary}{Corollary}

\theoremstyle{definition}
\newref{defn}{definition}{Definition}

\newcommand{\cat}[1]{\mathbf{#1}}
\renewcommand{\C}{\cat{C}}
\newcommand{\D}{\cat{D}}
\newcommand{\E}{\cat{E}}
\newcommand{\Set}{\cat{Set}}
\newcommand{\FinSet}{\cat{FinSet}}
\newcommand{\Grp}{\cat{Grp}}
\newcommand{\Ab}{\cat{Ab}}
\newcommand{\Ring}{\cat{Ring}}
\renewcommand{\Vec}{\cat{Vec}}
\newcommand{\Hask}{\cat{Hask}}
\newcommand{\Agda}{\cat{Agda}}
\newcommand{\Mat}{\cat{Mat}}
\newcommand{\Num}{\cat{Num}}

\newcommand{\pb}[1][dr]{\save*!/#1-1.2pc/#1:(-1,1)@^{|-}\restore}
\newcommand{\po}[1][dr]{\save*!/#1+1.2pc/#1:(1,-1)@^{|-}\restore}

\AtBeginSection[]
{
\begin{frame}[c,plain,noframenumbering]
\frametitle{План лекции}
\tableofcontents[currentsection]
\end{frame}
}

\makeatletter
\defbeamertemplate*{footline}{my theme}{
    \leavevmode
}
\makeatother

\begin{document}

\title{Теория категорий}
\subtitle{Функторы}
\author{Валерий Исаев}
\maketitle

\section{Определение}

\begin{frame}
\frametitle{Определение функторов}
\begin{itemize}
\item Функторы между категориями $\C$ и $\D$ -- это морфизмы категорий.
\item Функтор $F$ состоит из функции $F : Ob(\C) \to Ob(\D)$ и функций $F : Hom_\C(X, Y) \to Hom_\D(F(X), F(Y))$ для всех $X, Y \in Ob(\C)$.
\item Эти функции должны сохранять тождественные морфизмы и композиции:
\[ F(id_X) = id_{F(X)} \]
\[ F(g \circ f) = F(g) \circ F(f) \]
\end{itemize}
\end{frame}

\begin{frame}
\frametitle{Забывающие функторы}
\begin{itemize}
\item Забывающий функтор $\Grp \to \Set$, сопоставляющий каждой группе множество ее элементов.
\item Для других алгебраических структур тоже существуют забывающие функторы $\Ring \to \Set$, $\Ab \to \Set$, и так далее.
\item Можно задавать функторы, которые забывают не всю информацию.
\item Например, существует два забывающих функтора $\Ring \to \Grp$ и $\Ring \to \Ab$.
\end{itemize}
\end{frame}

\begin{frame}
\frametitle{Примеры функторов}
\begin{itemize}
\item Функторы между категориями предпорядков -- это в точности монотонные функции.
\item Если $M$ и $N$ -- пара моноидов, и $\C_M$ и $\C_N$ -- категории на одном объекте, соотетствующие этим моноидам, то функторы между $\C_M$ и $\C_N$ -- это в точности гомоморфизмы моноидов $M$ и $N$.
\item Пусть $\C$ -- декартова категория и $A$ -- объект $\C$, тогда $A \times - : \C \to \C$ -- функтор, сопоставляющий каждому объекту $B$ объект $A \times B$ и каждому морфизму $f : B \to B'$ морфизм $id_A \times f : A \times B \to A \times B'$.
\item Существует очевидный функтор $I : \Agda \to \Set$.
\item Функторам в агде соответствуют функторы $\Agda \to \Agda$.
\end{itemize}
\end{frame}

\begin{frame}
\frametitle{Функторы и дуальность}
\begin{itemize}
\item Каждому функтору $F : \C \to \D$ можно сопоставить функтор $F^{op} : \C^{op} \to \D^{op}$.
\item Другими словами существует биекция между множествами функторов $\C \to \D$ и $\C^{op} \to \D^{op}$.
\item С другой стороны, функторы вида $\C^{op} \to \D$ никак не связаны с функторами вида $\C \to \D$.
\item Первые называются контравариантными функторами, а вторые -- ковариантными.
\end{itemize}
\end{frame}

\begin{frame}
\frametitle{Пределы и копределы функторов}
\begin{itemize}
\item Для любого функтора $F : \cat{J} \to \C$ можно определить понятие предела $lim\ F$ и копредела $colim\ F$. Определение такое же как и для диаграмм.
\item Категории $\cat{J}$ можно рассматривать как обобщение графов, а функтор $F : \cat{J} \to \C$ -- как обощение диаграмм в $\C$.
\item Любой диаграмме можно сопоставить функтор, и наоборот. (Эти конструкции не взаимообратные)
\item Но пределы и копределы соответствующих диаграмм и функторов будут совпадать.
\item Функторы $F : \cat{J} \to \C$ тоже называют диаграммами.
\end{itemize}
\end{frame}

\section{Изоморфизм категорий}

\begin{frame}
\frametitle{Изоморфные категории}
\begin{itemize}
\item Для любой категории $\C$ существует тождественный функтор $Id_\C : \C \to \C$, отправляющий каждый объект и морфизм в себя.
\item Если $F : \C \to \D$ и $G : \D \to \E$, то функтор $G \circ F : \C \to \E$ определяется на объектах и на морфизмах как композиция $F$ и $G$.
\item Композиция функторов -- ассоциативна, тождественный функтор является единицей для композиции.
\item Функтор $F : \C \to \D$ называется \emph{изоморфизмом} категорий, если существует функтор $G : \D \to \C$ такой, что $G \circ F = Id_\C$ и $F \circ G = Id_\D$.
\item Категории $\C$ и $\D$ \emph{изоморфны}, если существует изоморфизм $F : \C \to \D$.
\end{itemize}
\end{frame}

\begin{frame}
\frametitle{Злые понятия}
\begin{itemize}
\item Как правило, имея две группы, не имеет смысла спрашивать равны ли они; нужно спрашивать об их изоморфности.
\item Это верно для объектов в любой категории.
\item Любое понятие, которое говорит о равенстве объектов некоторой категории, называют \emph{злым}.
\item Изоморфизм категорий -- злое понятие.
\end{itemize}
\end{frame}

\end{document}
