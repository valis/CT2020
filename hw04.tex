\documentclass[draft]{article}
\usepackage[russian]{babel}
\usepackage[utf8]{inputenc}
\usepackage{cmap}
\usepackage{amsfonts}
\usepackage[all]{xy}

\newcommand{\cat}[1]{\mathbf{#1}}
\renewcommand{\C}{\cat{C}}
\newcommand{\Set}{\cat{Set}}
\newcommand{\Grp}{\cat{Grp}}
\newcommand{\Ab}{\cat{Ab}}
\newcommand{\Vec}{\cat{Vec}}
\newcommand{\Hask}{\cat{Hask}}
\newcommand{\Mat}{\cat{Mat}}
\newcommand{\Num}{\cat{Num}}

\newcommand{\im}{\mathrm{Im}}
\newcommand{\bool}{\mathrm{Bool}}
\newcommand{\true}{\mathrm{true}}
\newcommand{\false}{\mathrm{false}}
\newcommand{\andb}{\mathrm{and}}
\newcommand{\orb}{\mathrm{or}}

\begin{document}

\title{Задания}
\maketitle

\begin{enumerate}

\item Опишите в категории (пред)порядка следующие конструкции:
\begin{enumerate}
\item Начальные объекты.
\item Копроизведения объектов.
\end{enumerate}

\item Докажите, что если $A \amalg B$ существует, то $B \amalg A$ тоже существует и изоморфен $A \amalg B$.

\item Начальный объект $0$ произвольной категории называется \emph{строгим}, если любой морфизм вида $X \to 0$ является изоморфизмом.
Например, в $\Set$ пустое множество является строгим начальным объектом.
В $\Grp$ тривиальная группа не является строгим начальным объектом, хоть и является начальным.

Докажите, что в произвольной категории начальный объект $0$ является строгим тогда и только тогда, когда для любого $X$ произведение $X \times 0$ существует и $X \times 0 \simeq 0$.

\item Докажите, что в $\Ab$ существуют все копроизведения.

\item Приведите нетривиальный пример категории, в которой для всех $A$ и $B$ существуют сумма и произведение и $A \amalg B \simeq A \times B$.

\item Идемпотентный морфизм $h : B \to B$ является расщепленным, если существуют $f : A \to B$ и $g : B \to A$ такие, что $g \circ f = id_A$ и $f \circ g = h$.
Докажите, что если в категории существуют коуравнители, то любой идемпотентный морфизм расщеплен.

\item При каких условиях в категории (пред)порядка существует булевский объект?

\item Пусть в категории $\C$ есть все конечные произведения и булевский объект.
Сконструируйте в $\C$ морфизмы $\andb, \orb : \bool \times \bool \to \bool$, такие что следующие диаграммы коммутируют
\[ \xymatrix{ 1 \ar[d]_{\langle \true, \true \rangle} \ar[rd]^{\true} \\
              \bool \times \bool \ar[r]_-{\andb} & \bool
            }
\qquad \xymatrix{ 1 \ar[d]_{\langle \true, \true \rangle} \ar[rd]^{\true} \\
              \bool \times \bool \ar[r]_-{\orb} & \bool
            } \]
\[ \xymatrix{ 1 \ar[d]_{\langle \true, \false \rangle} \ar[rd]^{\false} \\
              \bool \times \bool \ar[r]_-{\andb} & \bool
            }
\qquad \xymatrix{ 1 \ar[d]_{\langle \true, \false \rangle} \ar[rd]^{\true} \\
              \bool \times \bool \ar[r]_-{\orb} & \bool
            } \]
\[ \xymatrix{ 1 \ar[d]_{\langle \false, \true \rangle} \ar[rd]^{\false} \\
              \bool \times \bool \ar[r]_-{\andb} & \bool
            }
\qquad \xymatrix{ 1 \ar[d]_{\langle \false, \true \rangle} \ar[rd]^{\true} \\
              \bool \times \bool \ar[r]_-{\orb} & \bool
            } \]
\[ \xymatrix{ 1 \ar[d]_{\langle \false, \false \rangle} \ar[rd]^{\false} \\
              \bool \times \bool \ar[r]_-{\andb} & \bool
            }
\qquad \xymatrix{ 1 \ar[d]_{\langle \false, \false \rangle} \ar[rd]^{\false} \\
              \bool \times \bool \ar[r]_-{\orb} & \bool
            } \]

\item Мы видели, что объекты $2$ и $1$ могут быть изоморфны. Если $2$ является булевским объектом, то это все равно может произойти, но эту ситуацию легко отследить.

Пусть $\C$ -- категория с конечными произведениями.
Докажите, что следующие утверждения эквивалентны:
\begin{enumerate}
\item $\C$ -- категория предпорядка.
\item В $\C$ терминальный объект является булевским.
\item В $\C$ существует булевский объект, такой что $\true = \false$.
\end{enumerate}

\end{enumerate}

\end{document}
