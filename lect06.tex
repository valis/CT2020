\documentclass{beamer}

\usepackage[english,russian]{babel}
\usepackage[utf8]{inputenc}
\usepackage[all]{xy}
\usepackage{ifthen}
\usepackage{xargs}

\usetheme{Szeged}
% \usetheme{Montpellier}
% \usetheme{Malmoe}
% \usetheme{Berkeley}
% \usetheme{Hannover}
\usecolortheme{beaver}

\newcommand{\newref}[4][]{
\ifthenelse{\equal{#1}{}}{\newtheorem{h#2}[hthm]{#4}}{\newtheorem{h#2}{#4}[#1]}
\expandafter\newcommand\csname r#2\endcsname[1]{#4~\ref{#2:##1}}
\newenvironmentx{#2}[2][1=,2=]{
\ifthenelse{\equal{##2}{}}{\begin{h#2}}{\begin{h#2}[##2]}
\ifthenelse{\equal{##1}{}}{}{\label{#2:##1}}
}{\end{h#2}}
}

\newref[section]{thm}{theorem}{Theorem}
\newref{lem}{lemma}{Lemma}
\newref{prop}{proposition}{Proposition}
\newref{cor}{corollary}{Corollary}

\theoremstyle{definition}
\newref{defn}{definition}{Definition}

\newcommand{\cat}[1]{\mathbf{#1}}
\renewcommand{\C}{\cat{C}}
\newcommand{\D}{\cat{D}}
\newcommand{\E}{\cat{E}}
\newcommand{\Set}{\cat{Set}}
\newcommand{\Agda}{\cat{Agda}}
\newcommand{\FinSet}{\cat{FinSet}}
\newcommand{\Grp}{\cat{Grp}}
\newcommand{\Mon}{\cat{Mon}}
\newcommand{\CMon}{\cat{CMon}}
\newcommand{\Ab}{\cat{Ab}}
\newcommand{\Ring}{\cat{Ring}}
\renewcommand{\Vec}{\cat{Vec}}
\newcommand{\Hask}{\cat{Hask}}
\newcommand{\Mat}{\cat{Mat}}
\newcommand{\Num}{\cat{Num}}

\newcommand{\pb}[1][dr]{\save*!/#1-1.2pc/#1:(-1,1)@^{|-}\restore}
\newcommand{\po}[1][dr]{\save*!/#1+1.2pc/#1:(1,-1)@^{|-}\restore}

\AtBeginSection[]
{
\begin{frame}[c,plain,noframenumbering]
\frametitle{План лекции}
\tableofcontents[currentsection]
\end{frame}
}

\makeatletter
\defbeamertemplate*{footline}{my theme}{
    \leavevmode
}
\makeatother

\begin{document}

\title{Теория категорий}
\subtitle{Естественные преобразования}
\author{Валерий Исаев}
\maketitle

\section{Подкатегории}

\begin{frame}
\frametitle{Подкатегории}
\begin{itemize}
\item \emph{Подкатегория} $\C'$ категории $\C$ -- это подкласс объектов $\C$ и для каждой пары объектов $X,Y$ в $\C'$ подкласс множества $Hom_\C(X,Y)$ такие, что $\C'$ содержит тождественные морфизмы для любого $X \in \C'$ и замкнут относительно композиции.
\item Функтор $F : \C \to \D$ называется \emph{строгим} (\emph{faithful}), если для любых $X,Y \in Ob(\C)$ функция $F : Hom_\C(X,Y) \to Hom_\D(F(X),F(Y))$ инъективна.
\item Подкатегории категории $\C$ -- классы эквивалентности строгих инъективных на объектах функторов.
\end{itemize}
\end{frame}

\begin{frame}
\frametitle{Полные подкатегории}
\begin{itemize}
\item Подкатегория $\C$' категории $\C$ называется \emph{полной}, если для любых объектов $X,Y$ в $\C'$ множества $Hom_{\C'}(X,Y)$ и $Hom_\C(X,Y)$ равны.
\item Функтор $F : \C \to \D$ называется \emph{полным}, если для любых $X,Y \in Ob(\C)$ функция $F : Hom_\C(X,Y) \to Hom_\D(F(X),F(Y))$ сюръективна.
\item Полные подкатегории категории $\C$ -- классы эквивалентности полных строгих функторов.
\item Полные строгие функторы мы будем называть \emph{вложениями} категорий.
\end{itemize}
\end{frame}

\begin{frame}
\frametitle{Примеры}
\begin{itemize}
\item $\FinSet$ -- полная подкатегория $\Set$.
\item $\Ab$ -- полная подкатегория $\Grp$.
\item Категория частичных порядков -- полная подкатегория предпорядков.
\item Категория предпорядков вкладывается в категорию категорий.
\item Категория моноидов вкладывается в категорию категорий.
\item Существует не полное вложение $\Agda$ в $\Set$.
\item Все забывающий функторы, которые мы рассматривали, являются строгими.
\item Обратное тоже верно: любой строгий функтор является в некотором смысле забывающим.
\end{itemize}
\end{frame}

\section{Естественные преобразования}

\begin{frame}
\frametitle{Определение}
\begin{itemize}
\item Можно определить понятие морфизма между функторами $F, G : \C \to \D$.
\item \emph{Естественное преобразование} $\alpha : F \to G$ -- это функция, сопоставляющая каждому объекту $X$ из $\C$ морфизм $\alpha_X : F(X) \to G(X)$, удовлетворяющая условию, что для любого морфизма $f : X \to Y$ в $\C$ следующий квадрат коммутирует:
\[ \xymatrix{ F(X) \ar[r]^{\alpha_X} \ar[d]_{F(f)} & G(X) \ar[d]^{G(f)} \\
              F(Y) \ar[r]_{\alpha_Y}               & G(Y)
            } \]
\end{itemize}
\end{frame}

\begin{frame}
\frametitle{Категория функторов}
\begin{itemize}
\item Естественных преобразование отображает только объекты, но можно показать, что оно задает действие и на морфизмах.
\item Если $\alpha : F \to G$ -- естественное преобразование, то каждому морфизму $f : X \to Y$ в $\C$ можно сопоставить морфизм $\alpha_f : F(X) \to G(Y)$ в $\D$.
\item Морфизм $\alpha_f$ определяется как композиция $F(f) \circ \alpha_Y$, что равно $\alpha_X \circ G(f)$ по естественности.
\item Этот морфизм -- это диагональ в коммутативном квадрате, который появляется в определении естественности.
\end{itemize}
\end{frame}

\begin{frame}
\frametitle{Композиция естественных преобразований}
\begin{itemize}
\item Если $\alpha : F \to G$ и $\beta : G \to H$ -- пара естественных преобразований, то можно определить их композицию $\beta \circ \alpha : F \to H$ как функцию, сопоставляющую каждому $X$ из $\C$ морфизм $\beta_X \circ \alpha_X : F(X) \to H(X)$.
\item Композиция $\beta \circ \alpha$ -- естественна:
\[ \xymatrix{ F(X) \ar[r]^{\alpha_X} \ar[d]_{F(f)} & G(X) \ar[d]^{G(f)} \ar[r]^{\beta_X} & H(X) \ar[d]^{H(f)} \\
              F(Y) \ar[r]_{\alpha_Y}               & G(Y) \ar[r]_{\beta_Y}               & H(Y)
            } \]
\end{itemize}
\end{frame}

\begin{frame}
\frametitle{Категория функторов}
\begin{itemize}
\item Для любого функтора $F : \C \to \D$ существует тождественное естественное преобразование, сопоставляющее каждому $X$ тождественный морфизм $id_{F(X)} : F(X) \to F(X)$.
\item Композиция -- ассоциативна, тождественное преобразование является единицей для композиции.
\item Таким образом, для любой пары категорий $\C$ и $\D$ существует категория функторов, которая обозначается $\D^\C$.
\end{itemize}
\end{frame}

\begin{frame}
\frametitle{Эквивалентность категорий}
\begin{itemize}
\item Функтор $F : \C \to \D$ называются \emph{эквивалентностью} категорий, если существует функтор $G : \D \to \C$, такой что $G \circ F$ изоморфен $Id_\C$ в категории $\C^\C$, и $F \circ G$ изоморфен $Id_\D$ в категории $\D^\D$.
\item Категории $\C$ и $\D$ называются \emph{эквивалентными}, если существует эквивалентность $F : \C \to \D$.
\item Чтобы убедиться, что функтор является эквивалентностью, нужно проверять много условий.
\item Функтор $F : \C \to \D$ является эквивалентностью, если он полный, строгий и \emph{существенно сюръективен на объектах}.
\item Последнее условие означает, что для любого объекта $X$ в $\D$ существует объект $Y$ в $\C$, такой что $F(Y)$ изоморфен $X$.
\end{itemize}
\end{frame}

\begin{frame}
\frametitle{Пример}
\begin{itemize}
\item Определим функтор $F : \Mat \to \Vec$, такой что $F(n) = \mathbb{R}^n$ и $F(A)(v) = A \cdot v$.
\item Из линейной алгебры мы знаем, что между линейными операторами и матрицами есть биекция, которая описывается указаным выше способом.
\item Таким образом, этот функтор полный и строгий.
\item Из линейной алгебры мы знаем, что любое конечномерное векторное пространство $V$ изоморфно пространству $\mathbb{R}^{dim(V)}$.
\item Таким образом, $F$ -- существенно сюръективен на объектах, и, следовательно, является эквивалентностью.
\end{itemize}
\end{frame}

\begin{frame}
\frametitle{Доказательство}
\begin{prop}
Функтор является эквивалентностью тогда и только тогда, когда он полный, строгий и существенно сюръективен на объектах.
\end{prop}
\begin{proof}
Пусть $F : \C \to \D$ -- эквивалентность категорий.
Пусть $G : \D \to \C$ -- обратный к нему, $\alpha : G \circ F \simeq Id_\C$ и $\beta : F \circ G \simeq Id_\D$.

Тогда $F$ -- существенно сюръективен на объектах.
Действительно, для любого $X \in Ob(\D)$ возьмём $Y = G(X)$, тогда $\beta_X : F(G(X)) \simeq X$.

Покажем, что $F$ -- строгий.
Пусть $F(f) = F(f')$ для некоторых $f,f' : X \to Y$.
Тогда по естественности $\alpha$ получается, что $f = \alpha_Y \circ G(F(f)) \circ \alpha^{-1}_X = \alpha_Y \circ G(F(f')) \circ \alpha^{-1}_X = f'$.
Аналогично доказывается, что $G$ -- строгий.
\end{proof}
\end{frame}

\begin{frame}
\frametitle{Доказательство (продолжение)}
Докажем, что $F$ -- полный.
Пусть $h : F(X) \to F(Y)$ -- некоторая стрелка.
Тогда определим стрелку $f : X \to Y$ как следующую композицию:
\[ \xymatrix{ X \ar[r]^-{\alpha^{-1}_X} & G(F(X)) \ar[r]^{G(h)} & G(F(Y)) \ar[r]^-{\alpha_Y} & Y } \]
Тогда $\alpha^{-1}_Y \circ f \circ \alpha_X = G(h)$.
С другой стороны, по естественности $\alpha$ у нас есть равенство $\alpha^{-1}_Y \circ f \circ \alpha_X = G(F(f))$.
Следовательно $G(h) = G(F(f))$.
По строгости $G$ мы получаем, что $h = F(f)$.
Таким образом, $f$ -- прообраз $h$, то есть $F$ -- полный.
\end{frame}

\begin{frame}
\frametitle{Доказательство (в обратную сторону)}
Пусть $F$ -- полный, строгий и существенно сюръективен на объектах.
Тогда для любого $X \in Ob(\D)$ существует объект $Y \in Ob(\C)$ и изоморфизм $\alpha_X : F(Y) \simeq X$.
Определим $G : Ob(\D) \to Ob(\C)$ как функцию, возвращающую на каждом $X$ такой $Y$ (не важно какой конкретно).
\[ \xymatrix{ F(G(X)) \ar[r]^-{\alpha_X} \ar@{-->}[d]_{F(f')} & X \ar[d]^f \\
              F(G(Y)) \ar[r]_-{\alpha_Y}                      & Y
            } \]
Так как $F$ полон, то для каждого $f : X \to Y$ существует $f' : G(X) \to G(Y)$, такая что $F(f') = \alpha^{-1}_Y \circ f \circ \alpha_X$.
Так как $F$ строг, то такая стрелка уникальна.
Положим $G(f) = f'$.
Из уникальности $f'$ следует, что $G$ сохраняет $id$ и $\circ$.
\end{frame}

\begin{frame}
\frametitle{Доказательство (продолжение)}
Осталось проверить, что $G \circ F \simeq Id_\C$ и $F \circ G \simeq Id_\D$.
Преобразование $\alpha_X : F(G(X)) \simeq X$ естественно, так как коммутативный квадрат на предыдущем слайде -- в точности квадрат естественности $\alpha$.

Так как $F$ -- полный, то для любой стрелки $\alpha_{F(Y)} : F(G(F(Y))) \to F(Y)$ существует прообраз $\beta_Y : G(F(Y)) \to Y$.
Все $\beta_Y$ -- изоморфизмы, так как обратные к ним -- это прообразы $\alpha^{-1}_{F(Y)} : F(Y) \to F(G(F(Y)))$.
\end{frame}

\begin{frame}
\frametitle{Доказательство (окончание)}
Осталось проверить, что $\beta$ естественен.
\[ \xymatrix{ G(F(X)) \ar[r]^-{\beta_X} \ar[d]_{G(F(f))} & X \ar[d]^f \\
              G(F(Y)) \ar[r]_-{\beta_Y}                  & Y
            } \]
Применив $F$ к диаграмме выше, она начинает коммутировать, так как $\alpha$ естественен.
Но так как $F$ -- строгий, исходная диаграмма также коммутирует.
\end{frame}

\end{document}
