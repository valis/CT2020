\documentclass{beamer}

\usepackage[english,russian]{babel}
\usepackage[utf8]{inputenc}
\usepackage[all]{xy}
\usepackage{ifthen}
\usepackage{xargs}

\usetheme{Szeged}
% \usetheme{Montpellier}
% \usetheme{Malmoe}
% \usetheme{Berkeley}
% \usetheme{Hannover}
\usecolortheme{beaver}

\newcommand{\newref}[4][]{
\ifthenelse{\equal{#1}{}}{\newtheorem{h#2}[hthm]{#4}}{\newtheorem{h#2}{#4}[#1]}
\expandafter\newcommand\csname r#2\endcsname[1]{#4~\ref{#2:##1}}
\newenvironmentx{#2}[2][1=,2=]{
\ifthenelse{\equal{##2}{}}{\begin{h#2}}{\begin{h#2}[##2]}
\ifthenelse{\equal{##1}{}}{}{\label{#2:##1}}
}{\end{h#2}}
}

\newref[section]{thm}{theorem}{Theorem}
\newref{lem}{lemma}{Lemma}
\newref{prop}{proposition}{Proposition}
\newref{cor}{corollary}{Corollary}

\theoremstyle{definition}
\newref{defn}{definition}{Definition}

\newcommand{\cat}[1]{\mathbf{#1}}
\renewcommand{\C}{\cat{C}}
\newcommand{\D}{\cat{D}}
\newcommand{\E}{\cat{E}}
\newcommand{\Set}{\cat{Set}}
\newcommand{\FinSet}{\cat{FinSet}}
\newcommand{\Grp}{\cat{Grp}}
\newcommand{\Ab}{\cat{Ab}}
\newcommand{\Ring}{\cat{Ring}}
\renewcommand{\Vec}{\cat{Vec}}
\newcommand{\Hask}{\cat{Hask}}
\newcommand{\Agda}{\cat{Agda}}
\newcommand{\Mat}{\cat{Mat}}
\newcommand{\Num}{\cat{Num}}

\newcommand{\pb}[1][dr]{\save*!/#1-1.2pc/#1:(-1,1)@^{|-}\restore}
\newcommand{\po}[1][dr]{\save*!/#1+1.2pc/#1:(1,-1)@^{|-}\restore}

\AtBeginSection[]
{
\begin{frame}[c,plain,noframenumbering]
\frametitle{План лекции}
\tableofcontents[currentsection]
\end{frame}
}

\makeatletter
\defbeamertemplate*{footline}{my theme}{
    \leavevmode
}
\makeatother

\begin{document}

\title{Теория категорий}
\subtitle{$F$-алгебры}
\author{Валерий Исаев}
\maketitle

\section{(Ко)индуктивные типы данных}

\begin{frame}
\frametitle{Индуктивные типы данных}
\begin{itemize}
\item Допустим мы хотим описать объект в произвольной категории, являющийся аналогом какой-либо структуры данных (списки, деревья, и так далее).
\item Например, объект списков можно определить как бесконечную сумму
\[ 1 + A + A^2 + \ldots + A^n + \ldots \]
\item Но это определение плохо обобщается на другие типы данных.
\end{itemize}
\end{frame}

\begin{frame}
\frametitle{Индуктивные типы данных}
\begin{itemize}
\item Проблема в том, что для произвольного типа данных сложно описать "объект данного типа фиксированного размера".
\item Зато легко описать "объект размера $\leq$ данного".
\item Например, если $T_n$ -- объект бинарных деревьев высоты $\leq n$, то $1 + A \times T_n \times T_n$ -- объект бинарных деревьев высоты $\leq n + 1$.
\end{itemize}
\end{frame}

\begin{frame}
\frametitle{Индуктивные типы данных}
\begin{itemize}
\item Теперь, имея некоторый набор объектов $\{ T_n \}_{n \in \mathbb{N}}$, нам нужно собрать из них один объект.
\item В $\Set$ мы можем просто взять объединение множеств $\bigcup_{n \in \mathbb{N}} T_n$.
\item В теории категорий объединение заменяется копределом.
\item У нас есть очевидные морфизмы вложений $T_n \to T_{n+1}$. Теперь объект всех деревьев можно определить как следующий (секвенциальный) копредел:
\[ T_0 \to T_1 \to T_2 \to \ldots \]
\end{itemize}
\end{frame}

\begin{frame}
\frametitle{Общее определение индуктивных типов данных}
\begin{itemize}
\item Любой (ко)индуктивный тип данных можно задать в виде функтора $F : \C \to \C$.
\item Функтор, соответствующий, спискам определяется как $L_A(X) = 1 + A \times X$.
\item Функтор, соответствующий, бинарным деревьям определяется как $T_A(X) = 1 + A \times X \times X$.
\item В общем случае $F_D(X)$ задается как правая часть определения типа данных $D$, в котором все рекурсивные вхождения этого типа заменены на $X$.
\item Таким образом, если $X$ является интерпретацией $D$, то он должен удовлетворять уравнению $X \simeq F(X)$.
\end{itemize}
\end{frame}

\begin{frame}
\frametitle{Алгебры над эндофунктором}
\begin{itemize}
\item Существует два канонических способа найти решение уравнения $X \simeq F(X)$, как начальную $F$-алгебру или конечную $F$-коалгебру.
\item Если $F : \C \to \C$ -- некоторый эндофунктор, то \emph{$F$-алгебра} -- это пара $(X,\alpha)$, где $X$ -- объект $\C$, а $\alpha : F(X) \to X$ -- морфизм $\C$.
\item Морфизм $F$-алгебр $(X,\alpha)$ и $(Y,\beta)$ -- это морфизм $f : X \to Y$ в $\C$ такой, что следующая диаграмма коммутирует:
\[ \xymatrix{ F(X) \ar[r]^-\alpha \ar[d]_{F(f)} & X \ar[d]^f \\
              F(Y) \ar[r]_-\beta                & Y
            } \]
\end{itemize}
\end{frame}

\begin{frame}
\frametitle{Начальные алгебры}
\begin{itemize}
\item Легко видеть, что определения на предыдущем слайде задают категорию, которую мы будем обозначать $F\text{-}\mathrm{alg}$.
\item Начальный объект этой категории называется начальной $F$-алгеброй, и если она существует, то она является решением уравнения $X \simeq F(X)$.
\item Если функтор $F$ сохраняет секвенциальные копределы, то начальную $F$-алгебру можно определить как копредел следующей диаграммы:
\[ 0 \xrightarrow{!_{F(0)}} F(0) \xrightarrow{F(!_{F(0)})} F(F(0)) \xrightarrow{F(F(!_{F(0)}))} F(F(F(0))) \to \ldots \]
\end{itemize}
\end{frame}

\begin{frame}
\frametitle{Начальные алгебры и индуктивные типы}
\begin{itemize}
\item Начальные $F$-алгебры можно использовать для интерпретации индуктивных типов данных.
\item Если $N(X) = X \amalg 1$ -- функтор, соответствующий типу данных унарных натуральных чисел, то начальная $N$-алгебра -- это в точности объект натуральных чисел.
\item Если $L_A(X) = 1 \amalg A \times X$ -- функтор, соответствующий спискам, то начальная $L_A$-алгебра в $\Set$ -- это множество конечных списков элементов $A$.
\item В общем случае начальная $L_A$-алгебра -- это копроизведение объектов $1$, $A$, $A^2$, $A^3$, ...
\end{itemize}
\end{frame}

\begin{frame}
\frametitle{Коалгебры над эндофунктором}
\begin{itemize}
\item \emph{Коалгебры} над $F : \C \to \C$ определяются дуальным образом как пары $(X,\alpha)$, где $X$ -- объект $\C$, а $\beta : X \to F(X)$ -- морфизм $\C$.
\item По дуальности конечные $F$-коалгебры тоже являются решением уравнения $X \simeq F(X)$.
\item Если функтор $F$ сохраняет секвенциальные пределы, то конечную $F$-коалгебру можно определить как предел следующей диаграммы:
\[ \ldots \to F(F(F(1))) \xrightarrow{F(F(!_{F(1)}))} \to F(F(1)) \xrightarrow{F(!_{F(1)})} F(1) \xrightarrow{!_{F(1)}} 1 \]
\item Терминальные коалгебры являются интерпретацией коиндуктивных типов данных.
\item Например, терминальная $L_A$-коалгебра в $\Set$ -- это множество потенциально бесконечных списков.
\end{itemize}
\end{frame}

\end{document}
