\documentclass{beamer}

\usepackage[english,russian]{babel}
\usepackage[utf8]{inputenc}
\usepackage[all]{xy}
\usepackage{ifthen}
\usepackage{xargs}

\usetheme{Szeged}
% \usetheme{Montpellier}
% \usetheme{Malmoe}
% \usetheme{Berkeley}
% \usetheme{Hannover}
\usecolortheme{beaver}

\newcommand{\newref}[4][]{
\ifthenelse{\equal{#1}{}}{\newtheorem{h#2}[hthm]{#4}}{\newtheorem{h#2}{#4}[#1]}
\expandafter\newcommand\csname r#2\endcsname[1]{#4~\ref{#2:##1}}
\newenvironmentx{#2}[2][1=,2=]{
\ifthenelse{\equal{##2}{}}{\begin{h#2}}{\begin{h#2}[##2]}
\ifthenelse{\equal{##1}{}}{}{\label{#2:##1}}
}{\end{h#2}}
}

\newref[section]{thm}{theorem}{Theorem}
\newref{lem}{lemma}{Lemma}
\newref{prop}{proposition}{Proposition}
\newref{cor}{corollary}{Corollary}

\theoremstyle{definition}
\newref{defn}{definition}{Definition}

\newcommand{\cat}[1]{\mathbf{#1}}
\renewcommand{\C}{\cat{C}}
\newcommand{\Set}{\cat{Set}}
\newcommand{\Grp}{\cat{Grp}}
\newcommand{\Ab}{\cat{Ab}}
\newcommand{\Ring}{\cat{Ring}}
\renewcommand{\Vec}{\cat{Vec}}
\newcommand{\Hask}{\cat{Hask}}
\newcommand{\Mat}{\cat{Mat}}
\newcommand{\Num}{\cat{Num}}

\AtBeginSection[]
{
\begin{frame}[c,plain,noframenumbering]
\frametitle{План лекции}
\tableofcontents[currentsection]
\end{frame}
}

\makeatletter
\defbeamertemplate*{footline}{my theme}{
    \leavevmode
}
\makeatother

\begin{document}

\title{Теория категорий}
\subtitle{Конструкции в категориях}
\author{Валерий Исаев}
\maketitle

\section{Мономорфизмы и эпиморфизмы}

\begin{frame}
\frametitle{Мономорфизмы}
\begin{itemize}
\item В категории $\Set$ морфизм является изоморфизмом тогда и только тогда, когда он инъективен и сюръективен.
\item Верно ли это в произвольной категории?
\item Чтобы данный вопрос имел смысл, нам понадобится обощение понятия инъективных и сюръективных функций.
\item Морфизм $f : X \to Y$ называется \emph{мономорфизмом}, если для любых стрелок $g, h : Z \to X$ равенство $f \circ g = f \circ h$ влечет $g = h$.
\[ \xymatrix{ Z \ar@<-.5ex>[r]_h \ar@<.5ex>[r]^g & X \ar[r]^f & Y } \implies g = h \]
\item Мономорфизмы в $\Set$ -- это в точности инъективные функции.
\end{itemize}
\end{frame}

\begin{frame}
\frametitle{Мономорфизмы в алгебраических категориях}
\begin{prop}
В любой алгебраической категории ($\Grp$, $\Ab$, $\Ring$, \ldots) мономорфизмы -- это в точности инъективные функции.
\end{prop}
\begin{proof}
Докажем для $\Grp$, для остальных можно доказать аналогично.
Пусть $f : A \to B$ -- инъективный гомоморфизм групп, и $g,h : C \to A$ -- такие, что $f \circ g = f \circ h$.
Так как $f$ -- мономорфизм множеств, то $g$ и $h$ равны как функции над множествами.
Но отсюда следует, что они равны как гомоморфизмы групп.

Наоборот, если $f$ -- мономорфизм, то пусть $a_1, a_2 \in A$ такие, что $f(a_1) = f(a_2)$.
Тогда рассмотри пару гомоморфизмов $g_1, g_2 : \mathbb{Z} \to A$ таких, что $g_i(1) = a_i$.
Так как $f \circ g_1 = f \circ g_2$, то $g_1 = g_2$.
Следовательно $a_1 = g_1(1) = g_2(1) = a_2$.
\end{proof}
\end{frame}

\begin{frame}
\frametitle{Эпиморфизмы}
\begin{itemize}
\item Морфизм $f : X \to Y$ называется \emph{епиморфизмом}, если
\[ \xymatrix{ X \ar[r]^f & Y \ar@<-.5ex>[r]_h \ar@<.5ex>[r]^g & Z } \implies g = h \]
\item Эпиморфизмы в $\Set$ -- это в точности сюръективные функции.
\item Эпиморфизмы в категориях моноидов и колец не обязательно сюръективны.
\item Примеры: $\mathbb{N} \hookrightarrow \mathbb{Z}$, $\mathbb{Z} \hookrightarrow \mathbb{Q}$.
\end{itemize}
\end{frame}

\begin{frame}
\frametitle{Эпиморфизмы в $\Set$}
\begin{prop}
Эпиморфизмы в $\Set$ -- это в точности сюръективные функции.
\end{prop}
\begin{proof}
Пусть $f : A \to B$ -- сюръекция, и $g,h : B \to C$ -- такие, что $g \circ f = h \circ f$.
Тогда для любого $b \in B$ существует $a \in A$ такой, что $f(a) = b$.
Следовательно $g(b) = g(f(a)) = h(f(a)) = h(b)$.

Наоборот, если $f : A \to B$ -- эпиморфизм, то пусть $g,h : B \to \{ 0, 1 \}$ -- такие, что $g$ всегда равно 1, а $h(b)$ равно 1 в точности когда существует $a \in A$ такой, что $f(a) = b$.
Тогда $g \circ f = h \circ f$.
Следовательно $g = h$.
Следовательно для любого $b \in B$ верно, что $h(b) = g(b) = 1$, то есть существует $a \in A$ такой, что $f(a) = b$, то есть $f$ -- сюръекция.
\end{proof}
\end{frame}

\begin{frame}
\frametitle{Расщепленные моно- и эпиморфизмы}
\begin{itemize}
\item Морфизм $f : A \to B$ называется \emph{расщепленным мономорфизмом}, если существует $g : B \to A$ такой, что $g \circ f = id_A$.
\item Морфизм $g : B \to A$ называется \emph{расщепленным эпиморфизмом}, если существует $f : A \to B$ такой, что $g \circ f = id_A$.
\item Любой расщепленный мономорфизм является мономорфизмом. Действительно, если $f \circ h_1 = f \circ h_2$, то $h_1 = g \circ f \circ h_1 = g \circ f \circ h_2 = h_2$.
\item Любой расщепленный эпиморфизм является эпиморфизмом. Доказательство аналогично предыдущему.
\end{itemize}
\end{frame}

\begin{frame}
\frametitle{Сбалансированные категории}
\begin{itemize}
\item Категория называется сбалансированной, если любой мономорфный и эпиморфный морфизм является изоморфизмом.
\item Примеры сбалансированных категорий: $\Set$, $\Grp$, $\Ab$.
\item Примеры несбалансированных категорий: категории моноидов и колец.
\item Любой эпиморфный расщепленный мономорфизм является изоморфизмом.
Действительно, если $f : A \to B$ и $g : B \to A$ такие, что $g \circ f = id_A$, то $f \circ g \circ f = id_B \circ f$.
Следовательно $f \circ g = id_B$.
\item Любой мономорфный расщепленный эпиморфизм является изоморфизмом. Доказательство аналогично предыдущему.
\end{itemize}
\end{frame}

\section{Произведения}

\begin{frame}
\frametitle{Терминальные объекты}
\begin{itemize}
\item В категориях $\Set$ и $\Hask$ существует много похожих объектов: $\mathbb{Z}$ и $Integer$, $\{ * \}$ и $()$, $A \times B$ и $(a, b)$.
\item Существует ли обобщение этих конструкций в произвольных катгеориях?
\item Объект $A$ некоторой категории $\C$ называется \emph{терминальным}, если для любого объекта $B$ существует уникальная стрелка $B \to A$.
\item Другими словами, $A$ является терминальным, если для любого $B$ множество $Hom_\C(B,A)$ одноэлементно.
\end{itemize}
\end{frame}

\begin{frame}
\frametitle{Примеры терминальных объектов}
\begin{itemize}
\item В $\Set$ множество терминально тогда и только тогда, когда оно одноэлементно.
\item В $\Grp$ группа терминальна тогда и только тогда, когда она одноэлементна.
\item В $\Hask$ есть следующие терминальные объекты: $()$, $data\ Unit = Unit$.
\item Утверждение строчкой выше не является верным :(
\item В группоиде существует терминальный объект только если он тривиален.
\end{itemize}
\end{frame}

\begin{frame}
\frametitle{Уникальность терминальных объектов}
\begin{prop}
Любые два терминальных объекта изоморфны.
\end{prop}
\begin{proof}
Если $A$ и $B$ -- терминальные объекты, то существует пара стрелок $f : A \to B$ и $g : B \to A$.
При этом по уникальности верно, что $g \circ f = id_A$ и $f \circ g = id_B$.
\end{proof}

Терминальный объект обычно обозначают $1$.
Уникальный морфизм из $X$ в $1$ обычно обозначают $!_X : X \to 1$.
\end{frame}

\begin{frame}
\frametitle{Декартово произведение}
\begin{itemize}
\item Множество $B$ вместе с парой функций $\pi_i : B \to A_i$ является декартовым произведением множеств $A_1$ и $A_2$, если для любых $a_i \in A_i$ существует уникальный $b \in B$ такой, что $\pi_i(b) = a_i$.
\item Объект $B$ вместе с парой отображений $\pi_i : B \to A_i$ называется декартовым произведением $A_1$ и $A_2$, если для любых $f_i : C \to A_i$ существует уникальная стрелка $h : C \to B$ такая, что $\pi_i \circ h = f_i$.
\[ \xymatrix{     & C \ar[ld]_{f_1} \ar@{-->}[d]^h \ar[rd]^{f_2} & \\
              A_1 & B \ar[l]^{\pi_1} \ar[r]_{\pi_2}              & A_2
            } \]
\end{itemize}
\end{frame}

\begin{frame}
\frametitle{Уникальность декартова произведения}
\begin{prop}
Если ($B$, $\pi^B_i$) и ($C$, $\pi^C_i$) -- произведения объектов $A_1$ и $A_2$, то $B$ и $C$ изоморфны.
\end{prop}
\begin{proof}
По определению декартова произведения существуют стрелки $g : B \to C$ и $h : C \to B$ как на диаграмме ниже.
По уникальности $h \circ g = id_B$ и, аналогично, $g \circ h = id_C$.
\[ \xymatrix{     & B \ar[ld]_{\pi^B_1} \ar@{-->}[d]^g \ar[rd]^{\pi^B_2} & \\
              A_1 & C \ar[l]^{\pi^C_1} \ar@{-->}[d]^h \ar[r]_{\pi^C_2}   & A_2 \\
                  & B \ar[lu]^{\pi^B_1} \ar[ru]_{\pi^B_2}                &
            } \]
\end{proof}
\end{frame}

\begin{frame}
\frametitle{Произведение множества объектов}
\begin{itemize}
\item Если $\{ A_i \}_{i \in I}$ -- колекция объектов некоторой категории, то объект $B$ вместе с морфизмами $\pi_i : B \to A_i$ называется декартовым произведением объектов $A_i$, если для любой коллекции морфизмов $\{ f_i : C \to A_i \}_{i \in I}$ существует уникальная стрелка $h : C \to B$ такая, что $\pi_i \circ h = f_i$.
\item Декартово произведение объектов $\{ A_i \}_{i \in I}$ уникально с точностью до изоморфизма.
\item Оно обозначается $\prod\limits_{i \in I} A_i$.
Если $I = \{ 1, \ldots n \}$, то оно обозначается $A_1 \times \ldots \times A_n$.
Уникальный морфизм $C \to A_1 \times \ldots \times A_n$ обозначается $\langle f_1, \ldots f_n \rangle$.
\end{itemize}
\end{frame}

\begin{frame}
\frametitle{Декартовы категория}
Категория, в которой существует терминальный объект и бинарные произведения, называется \emph{декартовой}.

\begin{prop}
Категория декартова тогда и только тогда, когда в ней существуют все конечные произведения.
\end{prop}
\begin{proof}
Терминальный объект -- произведение пустого множества объектов, бинарные произведения -- произведение двух объектов.
И наоборот, произведение $A_i$ можно сконструировать как
\[ A_1 \times (A_2 \times \ldots (A_{n-1} \times A_n) \ldots) \]
Это можно доказать по индукции.
\end{proof}
\end{frame}

\end{document}
