\documentclass[draft]{article}
\usepackage[russian]{babel}
\usepackage[utf8]{inputenc}
\usepackage{cmap}
\usepackage{amsfonts}
\usepackage[all]{xy}

\newcommand{\cat}[1]{\mathbf{#1}}
\renewcommand{\C}{\cat{C}}
\newcommand{\Set}{\cat{Set}}
\newcommand{\Grp}{\cat{Grp}}
\newcommand{\Ab}{\cat{Ab}}
\newcommand{\Vec}{\cat{Vec}}
\newcommand{\Hask}{\cat{Hask}}
\newcommand{\Mat}{\cat{Mat}}
\newcommand{\Num}{\cat{Num}}

\newcommand{\im}{\mathrm{Im}}
\newcommand{\bool}{\mathrm{Bool}}
\newcommand{\true}{\mathrm{true}}
\newcommand{\false}{\mathrm{false}}
\newcommand{\andb}{\mathrm{and}}
\newcommand{\orb}{\mathrm{or}}

\begin{document}

\title{Задания}
\maketitle

\begin{enumerate}

\item Докажите, что два определения уравнителей, приводившихся в лекции, эквивалентны.

\item Морфизм $h : B \to B$ называется \emph{идемпотентным}, если $h \circ h = h$.
Докажите следующие факты:
\begin{enumerate}
\item Если $f : A \to B$ и $g : B \to A$ -- такие, что $g \circ f = id_A$, то $h = f \circ g$ является идемпотентным.
\item Если в категории есть уравнители, то обратное верно.
Конкретно, для любого идемпотентного морфизма $h : B \to B$ существуют $f : A \to B$ и $g : B \to A$ такие, что $g \circ f = id_A$ и $f \circ g = h$.
\end{enumerate}

\item Докажите, что любой расщепленный мономорфизм регулярен.

\item Мономорфизм $f : A \to B$ называется \emph{сильным}, если для любой коммутативного квадрата, где $e : C \to D$ является эпиморфизмом,
\[ \xymatrix{ C \ar[r] \ar[d]_e      & A \ar[d]^f \\
              D \ar[r] \ar@{-->}[ur] & B
            } \]
существует стрелка $D \to A$ такая, что диаграмма выше коммутирует.

Докажите, что любой регулярный мономорфизм силен.

\item Мономорфизм $f : A \to B$ называется \emph{экстремальным}, если для любого эпиморфизма $e : A \to C$ и любого морфизма $g : C \to B$ таких, что $g \circ e = f$, верно, что $e$ -- изоморфизм.

Докажите, что любой сильный мономорфизм экстремален.

\item Докажите, что если в категории все мономорфизмы регулярны, то она сбалансирована. Можно ли усилить это утверждение?

\item Докажите, что в $\Set$ все мономорфизмы регулярны.

\item Докажите, что в $\Ab$ все мономорфизмы регулярны.

\item Докажите следующие факты про пулбэки мономорфизмов:
\begin{enumerate}
\item Докажите, что пулбэк мономорфизма также является мономорфизмом.
\item Докажите, что пулбэк регулярного мономорфизма также является регулярным мономорфизмом.
\end{enumerate}

\item Докажите следующие факты про пулбэки эпиморфизмов:
\begin{enumerate}
\item Докажите, что пулбэк сюръективной функции в $\Set$ также является сюръективной функцией.
\item Докажите, что предыдущее утверждение не верно в категории моноидов для эпиморфизмов. Другими словами, необходимо привести пример эпиморфизма в категории моноидов, некоторый пулбэк которого не является эпиморфизмом.
\end{enumerate}

\item Пусть в диаграмме вида
\[ \xymatrix{ \bullet \ar[r] \ar[d] & \bullet \ar[r] \ar[d] & \bullet \ar[d] \\
              \bullet \ar[r]        & \bullet \ar[r]        & \bullet
            } \]
правый квадрат является пулбэком.
Докажите, что левый квадрат является пулбэком тогда и только тогда, когда внешний прямоугольник является пулбэком.

\item Пусть $f : A \to B$ и $g : B \to C$ -- морфизмы в некоторой категории, а $D \hookrightarrow C$ -- некоторый подобъект $C$.
Докажите, что $(g \circ f)^{-1}(D) \simeq f^{-1}(g^{-1}(D))$.

\item Докажите, что если в категории существуют терминальный объект и пулбэки, то в ней существуют все конечные пределы.

\end{enumerate}

\end{document}
