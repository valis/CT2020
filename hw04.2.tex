\documentclass[draft]{article}
\usepackage[russian]{babel}
\usepackage[utf8]{inputenc}
\usepackage{cmap}
\usepackage{amsfonts}
\usepackage{amsmath}
\usepackage[all]{xy}
\usepackage{bussproofs}

\newcommand{\cat}[1]{\mathbf{#1}}
\renewcommand{\C}{\cat{C}}
\newcommand{\Set}{\cat{Set}}
\newcommand{\Grp}{\cat{Grp}}
\newcommand{\Ab}{\cat{Ab}}
\newcommand{\Hask}{\cat{Hask}}
\newcommand{\Mat}{\cat{Mat}}
\newcommand{\Num}{\cat{Num}}

\newcommand{\im}{\mathrm{Im}}
\newcommand{\bool}{\mathrm{Bool}}
\newcommand{\true}{\mathrm{true}}
\newcommand{\false}{\mathrm{false}}
\newcommand{\andb}{\mathrm{and}}
\newcommand{\orb}{\mathrm{or}}
\newcommand{\inj}{\mathrm{inj}}

\newcommand{\ev}{\mathrm{ev}}
\newcommand{\zero}{\mathrm{zero}}
\newcommand{\suc}{\mathrm{suc}}
\newcommand{\rec}{\mathrm{rec}}

\begin{document}

\title{Задания}
\maketitle

\begin{enumerate}

\item Приведите пример нетривиальной категории порядка, являющейся декартово замкнутой.

\item Давайте докажем, что категории моноидов, групп и абелевых групп не являются декартово замкнутыми.
\begin{enumerate}
\item Докажите, что если в декартово замкнутой категории есть начальный объект, то он строгий.
\item Объект называется \emph{нулевым}, если он одновременно начальный и терминальный.
Докажите, что если в категории есть нулевой объект и начальный объект строгий, то эта категория тривиальна (то есть в ней между любой парой объектов существует уникальная стрелка).
\item Докажите, что в категориях, упомянутых в задании, есть нулевой объект и сделайте вывод, что они не декартово замкнуты.
\end{enumerate}

\item Докажите, что в любой декартово замкнутой категории $\C$ выполнены следующие утверждения:
\begin{enumerate}
\item Для любого объекта $A$ существует изоморфизм $A^1 \simeq A$.
\item Для любых объектов $A$, $B$ и $C$ существует изоморфизм $A^{B \times C} \simeq (A^B)^C$.
\item Умножение дистрибутивно над сложением, то есть для любых объектов $A$, $B$ и $C$ морфизм
\[ [\langle \pi_1, \inj_1 \circ \pi_2 \rangle, \langle \pi_1, \inj_2 \circ \pi_2 \rangle ] : (A \times B) \amalg (A \times C) \to A \times (B \amalg C) \]
является изоморфизмом, где $\inj_1 : B \to B \amalg C$ и $\inj_2 : C \to B \amalg C$ -- канонические морфизмы копроизведения, и
если $f : B \to X$, $g : C \to X$, то $[f,g] : B \amalg C \to X$ -- уникальный морфизм, удовлетворяющий $[f,g] \circ \inj_1 = f$ и $[f,g] \circ \inj_2$.
\item Если в $\C$ существует начальный объект 0, то для любого объекта $A$ существует изоморфизм $A^0 \simeq 1$.
\item Если в $\C$ существует копроизведение $B \amalg C$, то для любого объекта $A$ существует изоморфизм $A^{B \amalg C} \simeq A^B \times A^C$.
\end{enumerate}

\item Докажите, что в декартово замкнутой категории объект $2$ всегда является булевским.

\item Определите в произвольной декартово замкнутой категории комбинаторы $K$ и $S$, то есть следующие морфизмы:
\begin{align*}
K & : A \to A^B \\
S & : (C^B)^A \to (C^A)^{(B^A)}
\end{align*}

\item Одна из аксиом арифметики Пеано говорит, что функция $\suc$ должна быть инъективной.
Докажите, что в любой декартово замкнутой категории с объектом натуральных чисел морфизм $\suc$ является расщепленным мономорфизмом.

\item Одна из аксиом арифметики Пеано говорит, что ни для какого $x$ не верно, что $\zero = \suc(x)$.
В произвольной декартово замкнутой категории это может быть верно, но только если она является категорией предпорядка.
Докажите, что следующие утверждения эквивалентны.
\begin{enumerate}
\item $\C$ -- категория предпорядка.
\item В $\C$ терминальный объект является объектом натуральных чисел.
\item В $\C$ существует объект натуральных чисел, такой что для любого $x : 1 \to \mathbb{N}$ верно, что $\zero = \suc \circ x$.
\item В $\C$ существует объект натуральных чисел, такой что для некоторого $x : 1 \to \mathbb{N}$ верно, что $\zero = \suc \circ x$.
\end{enumerate}

\item Докажите, что если в декартово замкнутой категории существует все копроизведения, то в ней существует объект натуральных чисел.

\item Определите в произвольной декартово замкнутой категории с объектом натуральных чисел морфизм сложения $+ : \mathbb{N} \times \mathbb{N} \to \mathbb{N}$, удовлетворяющий следующим условиям:
\[ \xymatrix{ \mathbb{N} \ar[d]_{\langle \zero \circ !_\mathbb{N}, id_\mathbb{N} \rangle} \ar[rd]^{id_\mathbb{N}} \\
              \mathbb{N} \times \mathbb{N} \ar[r]_-{+} & \mathbb{N}
            }
\qquad
   \xymatrix{ \mathbb{N} \times \mathbb{N} \ar[r]^-{+} \ar[d]_{\suc \times id_\mathbb{N}} & \mathbb{N} \ar[d]^{\suc} \\
              \mathbb{N} \times \mathbb{N} \ar[r]_-{+} & \mathbb{N}
            } \]
Докажите, что сложение коммутативно и ассоциативно, то есть, что коммутируют следующие диаграммы:
\[ \xymatrix{ \mathbb{N} \times \mathbb{N} \ar[r]^{\langle \pi_2, \pi_1 \rangle} \ar[rd]_{+} & \mathbb{N} \times \mathbb{N} \ar[d]^{+} \\
                                                                           & \mathbb{N} \times \mathbb{N}
            } \]
\[ \xymatrix{ (\mathbb{N} \times \mathbb{N}) \times \mathbb{N} \ar[r]^{\simeq} \ar[d]_{+ \times id_\mathbb{N}} & \mathbb{N} \times (\mathbb{N} \times \mathbb{N}) \ar[r]^-{id_\mathbb{N} \times +} & \mathbb{N} \times \mathbb{N} \ar[d]^{+} \\
              \mathbb{N} \times \mathbb{N} \ar[rr]_{+} & & \mathbb{N}
            } \]

\end{enumerate}

\end{document}
