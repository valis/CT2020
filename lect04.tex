\documentclass{beamer}

\usepackage[english,russian]{babel}
\usepackage[utf8]{inputenc}
\usepackage[all]{xy}
\usepackage{ifthen}
\usepackage{xargs}

\usetheme{Szeged}
% \usetheme{Montpellier}
% \usetheme{Malmoe}
% \usetheme{Berkeley}
% \usetheme{Hannover}
\usecolortheme{beaver}

\newcommand{\newref}[4][]{
\ifthenelse{\equal{#1}{}}{\newtheorem{h#2}[hthm]{#4}}{\newtheorem{h#2}{#4}[#1]}
\expandafter\newcommand\csname r#2\endcsname[1]{#4~\ref{#2:##1}}
\newenvironmentx{#2}[2][1=,2=]{
\ifthenelse{\equal{##2}{}}{\begin{h#2}}{\begin{h#2}[##2]}
\ifthenelse{\equal{##1}{}}{}{\label{#2:##1}}
}{\end{h#2}}
}

\newref[section]{thm}{theorem}{Theorem}
\newref{lem}{lemma}{Lemma}
\newref{prop}{proposition}{Proposition}
\newref{cor}{corollary}{Corollary}

\theoremstyle{definition}
\newref{defn}{definition}{Definition}

\newcommand{\cat}[1]{\mathbf{#1}}
\renewcommand{\C}{\cat{C}}
\newcommand{\Set}{\cat{Set}}
\newcommand{\Grp}{\cat{Grp}}
\newcommand{\Ab}{\cat{Ab}}
\newcommand{\Ring}{\cat{Ring}}
\renewcommand{\Vec}{\cat{Vec}}
\newcommand{\Hask}{\cat{Hask}}
\newcommand{\Mat}{\cat{Mat}}
\newcommand{\Num}{\cat{Num}}

\newcommand{\im}{\mathrm{Im}}
\newcommand{\bool}{\mathrm{Bool}}
\newcommand{\true}{\mathrm{true}}
\newcommand{\false}{\mathrm{false}}

\newcommand{\ev}{\mathrm{ev}}
\newcommand{\zero}{\mathrm{zero}}
\newcommand{\suc}{\mathrm{suc}}
\newcommand{\fst}{\mathrm{fst}}
\newcommand{\snd}{\mathrm{snd}}
\newcommand{\unit}{\mathrm{unit}}

\newcommand{\pb}[1][dr]{\save*!/#1-1.2pc/#1:(-1,1)@^{|-}\restore}
\newcommand{\po}[1][dr]{\save*!/#1+1.2pc/#1:(1,-1)@^{|-}\restore}

\AtBeginSection[]
{
\begin{frame}[c,plain,noframenumbering]
\frametitle{План лекции}
\tableofcontents[currentsection]
\end{frame}
}

\makeatletter
\defbeamertemplate*{footline}{my theme}{
    \leavevmode
}
\makeatother

\begin{document}

\title{Теория категорий}
\subtitle{Копределы}
\author{Валерий Исаев}
\maketitle

\section{Копределы}

\begin{frame}
\frametitle{Дуальная категория}
Пусть $\C$ -- произвольная категория, тогда \emph{дуальная} ей категория $\C^{op}$ -- это категория, определяемая следующим образом:
\begin{itemize}
\item Объекты $\C^{op}$ совпадают с объектами $\C$.
\item Если $X$, $Y$ -- объекты $\C^{op}$, то $Hom_{\C^{op}}(X,Y)$ определяется как $Hom_\C(Y,X)$.
\item Композиция и тождественные морфизмы определяются так же, как в $\C$.
\end{itemize}
\end{frame}

\begin{frame}
\frametitle{Дуальность}
\begin{itemize}
\item В теории категорий зачастую определения и утверждения можно \emph{дуализировать}, применив их в дуальной категории.
\item Например, понятие эпиморфизма является дуальным к понятию мономорфизма.
\[ f \text{ -- моно: } \xymatrix{ Z \ar@<-.5ex>[r]_h \ar@<.5ex>[r]^g & X \ar[r]^f & Y } \implies g = h \]
\[ f \text{ -- эпи: } \xymatrix{ Z & X \ar@<-.5ex>[l]_g \ar@<.5ex>[l]^h & Y \ar[l]^f } \implies g = h \]
\item Часто к дуальным понятиям прибавляют приставку \emph{ко}. Например, эпиморфизмы можно называть комономорфизмами (или мономорфизмы можно называть коэпиморфизмами).
\end{itemize}
\end{frame}

\begin{frame}
\frametitle{Копределы}
\begin{itemize}
\item Копределы -- это дуальное понятие к понятию пределов.
\item \emph{Коконус} диаграммы $D$ -- это объект $A$ вместе с коллекцией морфизмов $a_v : D(v) \to A$ для каждой $v \in V$, удовлетворяющие условию, что для любого $e \in E$ следующая диаграмма коммутирует
\[ \xymatrix{ D(s(e)) \ar[rd]_{a_{s(e)}} \ar[r]^{D(e)} & D(t(e)) \ar[d]^{a_{t(e)}} \\
                                                       & A
            } \]
\end{itemize}
\end{frame}

\begin{frame}
\frametitle{Определение копределов}
\begin{itemize}
\item \emph{Копредел} диграммы $D$ -- это такой коконус $A$, что для любого коконуса $B$ существует уникальный морфизм $f : A \to B$, такой что для любой $v \in V$ следующая диаграмма коммутирует
\[ \xymatrix{ D(v) \ar[d]_{a_v} \ar[dr]^{b_v} \\
              A \ar[r]_f & B
            } \]
\item Копредел $D$ обозначается $colim\ D$.
\item Категория называется \emph{кополной} (\emph{конечно кополной}), если в ней существуют все малые (конечные) копределы.
\end{itemize}
\end{frame}

\begin{frame}
\frametitle{Уникальность копределов}
Дуализировать можно не только определения, но и утверждения.
\begin{prop}
Если $A$ и $B$ -- копределы диаграммы $D$, то существует изоморфизм $f : A \simeq B$, такой что $f \circ a_v = b_v$ для любой $v \in V$.
\end{prop}
\begin{proof}
Так как копредел в $\C$ -- это предел в $\C^{op}$, то это утверждение эквивалентно аналогичному утверждению для пределов.
\end{proof}
\end{frame}

\begin{frame}
\frametitle{Начальный объект}
\begin{itemize}
\item Объект называется \emph{начальным}, если он является копределом пустой диаграммы.
\item В $\Set$ существует единственный начальный объект -- пустое множество.
\item В $\Hask$ начальный объект -- пустой тип.
\item В $\Grp$ начальный объект -- тривиальная группа.
\end{itemize}
\end{frame}

\begin{frame}
\frametitle{Копроизведения объектов}
\begin{itemize}
\item \emph{Копроизведение} (\emph{сумма}) объектов $A_1$ и $A_2$ -- это копредел диаграммы $\quad A_1 \qquad A_2$. Копроизведение обозначается $A_1 \amalg A_2$ либо $A_1 + A_2$.
\item В $\Set$ копроизведение -- это размеченное объединение множеств.
\item В $\Hask$ копроизведение -- это $Either$.
\item В $\Grp$ копроизведение -- свободное произведение.
\end{itemize}
\end{frame}

\begin{frame}
\frametitle{Фактор-множества}
\begin{itemize}
\item Пусть $\sim$ -- отношение эквивалентности на множестве $B$.
\item Тогда можно определить множество $B / \sim$ классов эквивалентности элементов $B$ по этому отношению.
\item Существует каноническая функция $c : B \to B / \sim$, отправляющая каждый $b \in B$ в его класс эквивалентности.
\item Если рассматривать отношение $\sim$ как подмножество $B \times B$, то существуют проекции $f, g : \sim \to B$.
\item Стрелка $c$ уравнивает $f$ и $g$ и является универсальной с таким свойством.
\item Другими словами, $c$ является коуравнителем $f$ и $g$.
\end{itemize}
\end{frame}

\begin{frame}
\frametitle{Коуравнители}
\begin{itemize}
\item В произвольной категории коуравнители можно рассматривать как обобщение этой конструкции.
\item Пусть $B$ -- абелева группа, $A$ -- подгруппа $B$, $f : A \hookrightarrow B$ -- вложенние $A$ в $B$.
Тогда коядро $B/A$ -- это коуравнитель стрелок $f,0 : A \to B$.
\item И наоборот, коуравнитель стрелок $f,g : A \to B$ -- это коядро $B/\im(f-g)$.
\item \emph{Пушауты} -- дуальное понятие к понятию пулбэков.
\end{itemize}
\end{frame}

\section{Булевские объекты}

\begin{frame}
\frametitle{Копроизведение $1 \amalg 1$}
\begin{itemize}
\item В $\Set$ множество $\bool$ можно определить как копроизведение множеств $\{ \true \}$ и $\{ \false \}$, каждое из которых является терминальным.
\item Копроизведение $1 \amalg 1$ обычно обозначается как $2$.
\item Можно было бы в произвольной категории определить объект $\bool$ как копроизведение $1 \amalg 1$.
\item Но это недостаточно сильное определение. Мы не сможем никаких функций над ним определить.
\end{itemize}
\end{frame}

\begin{frame}
\frametitle{Булевский объект}
\begin{itemize}
\item Пусть в $\C$ существуют все конечные произведения.
\item Тогда \emph{булевский объект} в $\C$ -- это объект $\bool$ вместе с парой морфизмов $\true, \false : 1 \to \bool$, удовлетворяющий следующему условию.
\item Для любых $f, g : A \to B$ существует уникальная стрелка $h : \bool \times A \to B$, такая что
\[ \xymatrix{ A \ar[rr]^-{\langle \true \circ !_A, id_A \rangle} \ar[drr]_{f} & & \bool \times A \ar[d]^h \\
                                                                             & & B
            }
\qquad \xymatrix{ A \ar[rr]^-{\langle \false \circ !_A, id_A \rangle} \ar[drr]_{g} & & \bool \times A \ar[d]^h \\
                                                                              & & B
            }
\]
\end{itemize}
\end{frame}

\begin{frame}
\frametitle{Булевский объект и 2}
\begin{itemize}
\item Любой булевский объект является 2.
\item Действительно, если в определении булевского объекта в качестве $A$ взять 1, то мы получим в точности универсальное свойство $1 \amalg 1$.
\item Следовательно булевский объект уникален с точностью до изоморфизма.
\item Но не любой объект, являющийся 2, является булевским.
\item Действительно, в категории групп 2 изоморфен 1.
\item Но булевский объект изоморфен 1 только в категориях предпорядка.
\end{itemize}
\end{frame}

\begin{frame}
\frametitle{if}
\begin{itemize}
\item Мы можем сконструировать морфизм $if : \bool \times (C \times C) \to C$, удовлетворяющий
\[ \xymatrix{ C \times C \ar[d]_{\langle \true \circ !, id \rangle} \ar[rd]^{\pi_1} \\
              \bool \times (C \times C) \ar[r]_-{if} & C
            }
\quad \xymatrix{ C \times C \ar[d]_{\langle \false \circ !, id \rangle} \ar[rd]^{\pi_2} \\
              \bool \times (C \times C) \ar[r]_-{if} & C
            } \]
\item Действительно, в определении $\bool$ возьмем $A = C \times C$, $B = C$, $f = \pi_1$ и $g = \pi_2$.
\item Тогда существует уникальная стрелка $\bool \times (C \times C) \to C$, удовлетворяющая условиям выше.
\end{itemize}
\end{frame}

\end{document}
